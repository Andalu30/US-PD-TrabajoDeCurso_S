\documentclass[11pt]{article}
\usepackage[utf8]{inputenc}
\usepackage[spanish,es-tabla]{babel}
\usepackage[margin=35mm]{geometry} %Márgenes mas pequeños
\usepackage{cite}
\usepackage{amsmath,amssymb,amsfonts}
\usepackage{algorithmic}
\usepackage{graphicx}
\usepackage{textcomp}
\usepackage{xcolor}
\usepackage{booktabs}
\usepackage{optidef}
\usepackage{pdfpages}
\usepackage{float}
\restylefloat{table}
\usepackage{tocloft}
\renewcommand{\cftsecleader}{\cftdotfill{\cftdotsep}} %ToC con puntos en todos
\usepackage{minted}




\def\BibTeX{{\rm B\kern-.05em{\sc i\kern-.025em b}\kern-.08em
    T\kern-.1667em\lower.7ex\hbox{E}\kern-.125emX}}
%Lo de arriba son como imports, no le heches cuenta ;)

\title{
        \textbf{TweetKell}\\
        \medskip
        \large{Análisis de los datos privados de una cuenta de Twitter con Haskell}\\
        \bigskip
        Universidad de Sevilla\\ Ingeniería Informática Tecnologías Informáticas\\
        Programación Declarativa - Tercer curso}
        
\author{
Juan Arteaga Carmona\\
Herrera, Sevilla, España \\
\texttt{JuanArteaga@andalu30.me}\\
}



\begin{document}

%Portada con titulo e integrantes
\maketitle
\newpage

%Índice de la memoria
\tableofcontents
\listoftables
\listoffigures
\newpage

%Documento----
\section{Introducción}
A continuación se presenta la documentación realizada para la entrega final de la convocatoria de Septiembre del trabajo de curso de la asignatura Programación Declarativa del grado de Ingeniería Informática - Tecnologías Informáticas de la Universidad de Sevilla.

\section{Expectativas}
Al comienzo del trabajo se pretendía obtener una versión final que cumpliese con los requisitos expuestos por nuestro profesor, los cuales se pueden observar en el apartado \ref{apt:criterios}.

\section{Criterios de evaluación}\label{apt:criterios}
A continuación se muestran los criterios de evaluación a los que se someterá este trabajo, de acuerdo con la información que se encontraba en el aula virtual de la asignatura a día 2 de septiembre de 2019.

\begin{itemize}
    \item La fecha de entrega del trabajo es el día 8 de septiembre hasta media noche. La defensa del trabajo tendrá lugar la semana siguiente, del 9 al 13 (se anunciará por este canal las fechas).
    \item El trabajo debe ser un programa de temática "libre" (dentro de los ofertados en la convocatoria de febrero, o los que se haya hablado con los profesores previamente), totalmente funcional. Para la superación del trabajo, éste debe contener como mínimo:
    
    \begin{itemize}
        \item Dos aplicaciones de cada concepto básico de programación funcional visto en cada tema del primer bloque de la asignatura (temas 2,4,5,6,7,8). Es decir: al menos usar 2 funciones básicas de haskell, definir 2 funciones recursivas, 2 funciones por patrones, 2 usos de guardas, 2 usos de listas por comprensión, 2 usos de orden superior, 2 usos de tipados, etc.
    \item Creación de un módulo
    \item Creación de un tipo de datos nuevo y uso de éste.
    \item Uso de al menos dos de los conceptos avanzados y librerías vistos en los temas 3, 9, 11, 12, 13, 14 o 15 (deben ser de temas distintos).
    \end{itemize}
    
    
    \item La entrega debe contener, en un archivo comprimido (zip, máximo 50 mb):
    \begin{itemize}
        \item El código fuente Haskell (incluyendo un pequeño readme explicando el uso del programa).
        \item Un documento (formato libre) explicando la estructura del código y cómo se ha aplicado cada uno de los conceptos.
    \end{itemize}


\end{itemize}

\section{Elección del tipo de trabajo}
Se decidió que el trabajo fuese uno completamente diferente al que se propuso en la primera convocatoria \cite{trabajoPrimeraConv} ya que, tal y como se explicó en las conclusiones de este, elaborar un juego con Haskell y CodeWord no parecía ser una tarea sencilla ni interesante. 

Así pues, este trabajo se centrara en la temática de Data Science y análisis de datos, con especial énfasis en la recolección de los datos a través de la librería Aeson \cite{Aeson} y en la interactividad del programa como se verá en el apartado \ref{aplicacionConceptos}.


\section{Haskell}
Para este proyecto utilizaremos en lenguaje de programación Haskell\cite{haskellweb}.
Haskell es un lenguaje de programación puramente funcional creado por Paul Hodax en 1990 \cite{origenhaskell}.


\section{Datos de Twitter}
Al utilizar esta aplicación podremos extraer y consultar los datos de una cuenta de Twitter, para esto es necesario que dispongamos de los datos en un formato que nuestro PC pueda entender.

Para conseguir los datos deberemos de hacer a Twitter una petición para que nos permita descargarlos. Esta petición se hace a traves de los ajustes de privacidad en nuestro perfil y, dependiendo del tamaño de nuestra cuenta, en unos minutos se nos permitirá acceder a ellos. \cite{twitterData}

Una vez descargados, tendremos que descomprimirlos y ya podremos utilizar la aplicación.

\section{Implementación}
\subsection{Estructura del código}
El código se encuentra estructurado en varios módulos que, además de ser un requisito para la entrega del trabajo, permiten una mayor comodidad y simplicidad a la hora de desarrollar la aplicación.
A continuación se describen los módulos:
 \begin{itemize}
     \item \mint{haskell} TweeKell.hs
     Archivo principal del programa, contiene una función main que llama al menu principal del módulo MenuInteractivo.
     
     \item \mint{haskell} MenuInteractivo.hs
     Módulo encargado de generar los menús interactivos del programa además de recibir las instrucciones del usuario y llamar a la lógica correcta.
     
     \item \mint{haskell} TiposDatos.hs
     Módulo encargado de definir los tipos de datos que se utilizan para decodificar los archivos JSON y mostrar la información.
     
     \item \mint{haskell} TiposDatos2.hs
     Igual que TiposDatos, con la única diferencia de que este contiene las definiciones de los tipos de datos que producen alguna colisión en caso de que se encontrasen en TiposDatos.
     
     \item \mint{haskell} ParsersArchivos.hs
     Módulo encargado de parsear los archivos JSON de Twitter y de crear los objetos de los tipos que se definen en TiposDatos y TiposDatos2.
     
     \item \mint{haskell} AnalisisSimples.hs
     Módulo encargado de recibir el parseado realizado por ParsersArchivos y analizar y mostrar la información que el usuario ha solicitado para todos los tipos de análisis excepto para los que dependen de Tweets.
     
     \item \mint{haskell} AnalisisTweets.hs
     Igual que AnalisisSimples pero centrado en los que dependen del tipo Tweet.
 \end{itemize}


\subsection{Aplicación de los conceptos y ajuste a los criterios de evaluación}\label{aplicacionConceptos}

\begin{itemize}
    \item Aplicaciones de conceptos básicos de programación funcional.
    \begin{itemize}
        \item 2 funciones básicas de Haskell
        \item 2 funciones recursivas
        \item 2 funciones por patrones
        \item *2 funciones lambda
        \item *2 listas por compresión
        \item *2 funciones orden superior
        \item 2 usos de guardas
        \item 2 usos de tipados
        \item 
    \end{itemize}
    
    \item Creación de al menos un módulo
    \item Creación de un tipo de datos y uso de este
    \item Uso de dos conceptos avanzados
    \begin{itemize}
        \item Programa interactivo:
        \item Procesamiento JSON con Aeson:
    \end{itemize}
    
\end{itemize}



%Bibliografía
\begin{thebibliography}{99}

\bibitem{trabajoPrimeraConv} Juan Arteaga Carmona \textit{Recreando un juego interactivo con Haskell y CodeWorld-API} \href{https://github.com/Andalu30/US-PD-TrabajoDeCurso}{github.com/Andalu30/US-PD-TrabajoDeCurso - Repositorio privado en GitHub}

\bibitem{Aeson} Bryan O'Sullivan. \textit{Aeson: A fast Haskell JSON library} \href{https://github.com/bos/aeson}{github.com/bos/aeson - Repositorio en GitHub}

\bibitem{haskellweb} Paul Hudak, \textit{Haskell} \href{https://www.haskell.org/}{ Web del lenguaje}.

\bibitem{origenhaskell} Hudak, Paul; Hughes, John; Peyton Jones, Simon; Wadler, Philip (2007), \textit{A History of Haskell: Being Lazy with Class} ISBN 978-1-59593-766-7.

\bibitem{twitterData} Twitter \textit{Cómo acceder a tus datos de Twitter} \href{https://help.twitter.com/es/managing-your-account/accessing-your-twitter-data}{Enlace a la web de asistencia de Twitter}







\end{thebibliography}


\section*{Anexo}



\end{document}
